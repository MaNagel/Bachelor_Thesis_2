\chapter{Monotop Topologies in Extensions of the Standard Model}

\section{Brief Introduction to the Standard Model of Particle Physics}
The Standard Model of Particle Physics is an extensive Theory, describing the known elementary particles and their fundamental interactions.
In the last 40 years the Standard Model has been continuously developed and has withstand many tests on its consistency.
Nevertheless there are physical phenomena which can not be described, as there is for example the gravitation.
The fundamental characteristics of elementary particles described by the Standard Model are mass, spin, electric charge for the electromagnetic interaction and color charge for the strong interaction.
The Standard Model includes two different types of elementary fermions as there are the leptons and the quarks containing each three particle generations.
The third type of particles are the gauge bosons.
The gauge bosons are mediating the fundamental interactions between the elementary particles.
The electromagnetic interaction is mediated by the photons, while the strong interaction is given by the gluon.
As the Leptons are not color charged they are excluded by the strong interaction.
The third fundamental interaction is the weak force which is given by the $\mathrm{Z^0}$, $\mathrm{W^+}$ and $\mathrm{W^-}$.
In contrast to the photon and the gluon, the gauge bosons for the weak interaction are massive.
In Addition there is the higgs boson with a spin 0 which is connected to the higgs mechanism which explaining the mass for the fermions and weak gauge bosons.
An Overview to the Standard Model particles is shown in table \ref{StandardModel}.

\begin{table}
\centering
\begin{tabular}{|c c c |c c c c|}
	\hline 
	quarks &  &  & bosons &  &  &  \\ 
	\hline 
	$
	\left(\begin{array}{c}
		u \\ 
		d 
	\end{array}\right)
	$
	 & 
	 $
	 \left(\begin{array}{c}
	 c \\ 
	 s 
	 \end{array}\right)
	 $&
	 $
	 \left(\begin{array}{c}
	 t \\ 
	 b 
	 \end{array}\right)
	 $  &  $g$  & $\gamma$ & $Z^0$, $W^{\pm}$  & $H$  \\ 
	\hline 
	leptons &  &  & & & &  \\ 
	
	\hline 
	$
	\left(\begin{array}{c}
	e \\ 
	\nu_e 
	\end{array}\right)
	$
	& 
	$
	\left(\begin{array}{c}
	\mu \\ 
	\nu_\mu 
	\end{array}\right)
	$&
	$
	\left(\begin{array}{c}
	\tau \\ 
	\nu_\tau 
	\end{array}\right)
	$  & $\gamma$ & $Z^0$, $W^{\pm}$ & $H$ &  \\ 
	\hline 
\end{tabular}
\caption{Overview to the particles contained in the standard model.
	On the right of the table, there are the bosons interacting with the particles on the left.}
\label{StandardModel}
\end{table}


\section{Monotop Topologies in Proton-Proton Collisions}

In Theory there are many predictions for new kinds of particles.
One of these new particles types of particles are the so called "Vector Like Quarks" .
These are heavy particles with special properties for strong interaction and higgs mechanism.
In the following there will be considered more special processes for vector like quarks.
Those processes are "montop" events, where a vector like quark is produced and then decaying into one single top quark and another invisible final state.
In this context "invisible" means that there are particles created which are not detectable and so cause large amounts of "Missing Transverse Energy" $E_{\text{T}}^{\text{miss}}$.
The $E_{\text{T}}^{\text{miss}}$ is the energy which is needed do fulfil the momentum conservation in the transverse plane.
The properties of $E_{\text{T}}^{\text{miss}}$ will be further explained in chapter 2.
In figure \ref{} is shown an exemplary feynman graph for a monotop process, where the vector like quark decays into a top quark and a $\mathrm{Z^0}$ boson.
The $\mathrm{Z^0}$ by itself then decays into two neutrinos which are responsible for the  $E_{\text{T}}^{\text{miss}}$.